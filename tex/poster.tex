%==============================================================================
%== template for LATEX poster =================================================
%==============================================================================
%
%--A0 beamer slide-------------------------------------------------------------
\documentclass[5pt, final]{beamer}
\usepackage[orientation=portrait, size=a0,
            scale=1.25         % font scale factor
           ]{beamerposter}
           
\geometry{
  hmargin=2.5cm, % little modification of margins
}

%
\usepackage[utf8]{inputenc}

\linespread{1.05}
%
%==The poster style============================================================
\usetheme{sharelatex}

%==Title, date and authors of the poster=======================================
\title
[Agile Vorgehensmodelle, Master, WS 202425] % Conference
{ % Poster title
	Die Rolle des Scrum Masters: Moderation, Konfliktlösung und Teamförderung
}

\author{ % Authors
	Nico Riedlinger\inst{1}, Marc Weiss\inst{1}
}
\institute
[Very Large University] % General University
{
	\inst{1} HTWG Konstanz
}
\date{\today}



\begin{document}
	\begin{frame}[t]
		%==============================================================================
		\begin{multicols}{3}
			%==============================================================================
			%==The poster content==========================================================
			%==============================================================================
			
			\section{Aufgabenbereich}
			
%			In Ref.~\cite{ref1}...
%			In Refs.~\cite{ref1,ref2}...
%			On webpage~\cite{web}...
			Der Aufgabenbereich eines Scrum Masters ist sehr vielseitig.
			Als Teil des Teams übernimmt er eine koordinierende Rolle und ist Ansprechpartner in Problemsituationen \cite{meindl12}.
			Diese können bezogen auf die technische Seite, zum Beispiel mit dem zu entwickelnden Produkt, oder auf sozialer Ebene sein.
			Bei Unstimmigkeiten im Team ist der Scrum Master dafür verantwortlich, diese Konflikte zu beseitigen.
			
			Eine Einordnung des Scrum Masters innerhalb des Teams ist in Figure \ref{fig:sm-team} zu sehen.
			Hier bildet er eine Schnittstelle vom Team nach außen gegenüber Managern.
			
			\vskip1ex
			\begin{figure}
				\centering
				\includegraphics[width=0.95\columnwidth]{scrummaster}
				\caption{Einordnung des Scrum Masters innerhalb des gesamten Teams. Quelle: \cite{meindl12}}\label{fig:sm-team}
			\end{figure}
			\vskip2ex
			
			Des Weiteren ist er für die Organisation und Moderation von Besprechungen im Scrum-Kontext zuständig.
			Das umfasst sowohl das Sprint Planning als auch die Sprint Review sowie Retrospektive (kurz: Retro).
			Er verteilt im Voraus die jeweilige Agenda und moderiert die einzelnen Besprechungen.
			Letzteres gilt außerdem für das Daily Scrum.
			
			\subsection{Konfliktlösung}
			
			\subsection{Moderation}
			
			Der Scrum Master moderiert die Besprechungen, um eine hohe Effektivität und das Erreichen der Ziele sicherzustellen \cite{vantighem24}.
			Unter korrektem Einsatz verschiedener Techniken kann die Qualität von Besprechungen wie auch die Motivation der Teilnehmer für eine solche Besprechung erheblich gesteigert werden.
			
			Eine der Techniken ist die Erstellung und Überprüfung von Teamregeln.
			Ähnlich wie die \textit{Definition of Done} oder \textit{Definition of Ready} für das Erstellen und Abschließen von User Storys, handelt es sich hierbei um einen dynamischen Regelsatz.
			Dieser kann je nach Aufgaben und Anforderungen des Teams in gemeinsamer Absprache angepasst werden.
			Der Scrum Master beobachtet das Team während Besprechungen und weißt, ähnlich wie ein Schiedsrichter, auf Verstöße gegen die Regeln hin.
			Bei zu vielen Verstößen muss ein Grund dafür mit dem gesamten Team gefunden und gegebenenfalls das Regelwerk angepasst werden.
			
			Moderation unabhängig von Führungsrolle durchführen, um Interessenskonflikte zu vermeiden \cite[S. 23]{malten24}.
			
			\subsection{Teamförderung}
			
			\section{Abgrenzung zum Product Owner}
			
			Scrum Master vs. Product Owner \cite{me-company}.
			
			\section{Result}
			
			Steigerung der Produktivität des Teams \cite{vantighem24}.
			
			Abschirmung nach Außen vor zum Beispiel Management und Organisation der Anforderungen mit dem PO \cite{meindl12}.
			
%			\vskip1ex
%			\begin{table}
%				\centering
%				\caption{This is a table with scientific results.}
%				\begin{tabular}{ccccc}
%					\hline\hline
%					1 & 2 & 3 & 4 & 5\\
%					\hline
%					aaa & bbb & ccc & ddd & eee\\
%					aaaa & bbbb & cccc & dddd & eeee\\
%					aaaaa & bbbbb & ccccc & ddddd & eeeee\\
%					aaaaaa & bbbbbb & cccccc & dddddd & eeeeee\\
%					1.000 & 2.000 & 3.000 & 4.000 & 5.000\\
%					\hline\hline
%				\end{tabular}
%			\end{table}
%			\vskip2ex
			
			%==============================================================================
			%==End of content==============================================================
			%==============================================================================
			
			%--References------------------------------------------------------------------
			
			\subsection{References}
						
			\begin{thebibliography}{99}
				
%				\bibitem{ref1} J.~Doe, Article name, \textit{Phys. Rev. Lett.}
%				
%				\bibitem{ref2} J.~Doe, J.~Smith, Other article name, \textit{Phys. Rev. Lett.}
%				
%				\bibitem{web} \url{http://www.google.pl}
				
				\bibitem{malten24} M.~Malten, Effektive Team-Meetings: Impulse aus der agilen Praxis für bessere Besprechungen und Kommunikation
				
				\bibitem{meindl12} C.~Meindl, Der ScrumMaster, \url{https://alphanodes.com/de/der-scrummaster} (zul. besucht am 10.12.2024)
				
				\bibitem{me-company} Me~\&~Company~GmbH, Scrum Master: Coach, Moderator und Unterstützer, \url{https://www.me-company.de/magazin/scrum-master/} (zul. besucht am 10.12.2024)
				
				\bibitem{vantighem24} C.~Vantighem, Der Scrum Master: Definition, Aufgaben und Nutzen, \url{https://www.teammeter.com/de/scrum-master-aufgaben-verantwortung-und-nutzen} (zul. besucht am 10.12.2024)
				
			\end{thebibliography}
			%--End of references-----------------------------------------------------------
			
		\end{multicols}
		
		%==============================================================================
	\end{frame}
\end{document}
